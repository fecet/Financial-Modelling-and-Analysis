% Options for packages loaded elsewhere
\PassOptionsToPackage{unicode}{hyperref}
\PassOptionsToPackage{hyphens}{url}
%
\documentclass[
  12pt,
  oneside]{book}
\usepackage{amsmath,amssymb}
\usepackage{lmodern}
\usepackage{iftex}
\ifPDFTeX
  \usepackage[T1]{fontenc}
  \usepackage[utf8]{inputenc}
  \usepackage{textcomp} % provide euro and other symbols
\else % if luatex or xetex
  \usepackage{unicode-math}
  \defaultfontfeatures{Scale=MatchLowercase}
  \defaultfontfeatures[\rmfamily]{Ligatures=TeX,Scale=1}
\fi
% Use upquote if available, for straight quotes in verbatim environments
\IfFileExists{upquote.sty}{\usepackage{upquote}}{}
\IfFileExists{microtype.sty}{% use microtype if available
  \usepackage[]{microtype}
  \UseMicrotypeSet[protrusion]{basicmath} % disable protrusion for tt fonts
}{}
\makeatletter
\@ifundefined{KOMAClassName}{% if non-KOMA class
  \IfFileExists{parskip.sty}{%
    \usepackage{parskip}
  }{% else
    \setlength{\parindent}{0pt}
    \setlength{\parskip}{6pt plus 2pt minus 1pt}}
}{% if KOMA class
  \KOMAoptions{parskip=half}}
\makeatother
\usepackage{xcolor}
\IfFileExists{xurl.sty}{\usepackage{xurl}}{} % add URL line breaks if available
\IfFileExists{bookmark.sty}{\usepackage{bookmark}}{\usepackage{hyperref}}
\hypersetup{
  pdftitle={Financial Modeling Analysis},
  pdfauthor={Xie Zejian},
  hidelinks,
  pdfcreator={LaTeX via pandoc}}
\urlstyle{same} % disable monospaced font for URLs
\usepackage[left=0.8in, right=0.4in, top=0.5in, bottom=1in]{geometry}
\usepackage{longtable,booktabs,array}
\usepackage{calc} % for calculating minipage widths
% Correct order of tables after \paragraph or \subparagraph
\usepackage{etoolbox}
\makeatletter
\patchcmd\longtable{\par}{\if@noskipsec\mbox{}\fi\par}{}{}
\makeatother
% Allow footnotes in longtable head/foot
\IfFileExists{footnotehyper.sty}{\usepackage{footnotehyper}}{\usepackage{footnote}}
\makesavenoteenv{longtable}
\usepackage{graphicx}
\makeatletter
\def\maxwidth{\ifdim\Gin@nat@width>\linewidth\linewidth\else\Gin@nat@width\fi}
\def\maxheight{\ifdim\Gin@nat@height>\textheight\textheight\else\Gin@nat@height\fi}
\makeatother
% Scale images if necessary, so that they will not overflow the page
% margins by default, and it is still possible to overwrite the defaults
% using explicit options in \includegraphics[width, height, ...]{}
\setkeys{Gin}{width=\maxwidth,height=\maxheight,keepaspectratio}
% Set default figure placement to htbp
\makeatletter
\def\fps@figure{htbp}
\makeatother
\setlength{\emergencystretch}{3em} % prevent overfull lines
\providecommand{\tightlist}{%
  \setlength{\itemsep}{0pt}\setlength{\parskip}{0pt}}
\setcounter{secnumdepth}{5}
%\setmathfont{texgyrepagella-math.otf}
%\setmathfont{XITSMath-Regular.otf}

\usepackage{booktabs}
\usepackage{graphicx}
\usepackage{subfigure}
\usepackage{amssymb}
\usepackage{mathtools}
\usepackage{upgreek}
\usepackage{float}
\usepackage{extarrows}
\usepackage{longtable}
\usepackage{tikz}
\usepackage{commutative-diagrams}
\usepackage{letltxmacro}

\DeclareMathOperator{\tr}{Tr}
\DeclareMathOperator{\rank}{rank}
\DeclareMathOperator{\deter}{det}
\DeclareMathOperator{\diag}{diag}
\DeclareMathOperator{\eig}{eig}
\DeclareMathOperator{\vect}{vec}

\let\emptyset\varnothing


% \renewcommand{\mathbf}[1]{\symbf{#1}}
\newcommand{\bm}[1]{\symbf{#1}}

\newcommand{\lt}{<}
\newcommand{\gt}{>}
\newcommand{\R}{\mathbb{R}}
\newcommand{\Reals}{\mathbb{R}}
\newcommand{\Real}{\mathbb{R}}
\newcommand{\N}{\mathbb{N}}
\newcommand{\Q}{\mathbb{Q}}
\newcommand{\sub}{\subset}
\newcommand{\subsets}{\subset}
\newcommand{\exist}{\exists}
\DeclareMathOperator*{\argmax}{arg\,max}
\DeclareMathOperator*{\argmin}{arg\,min}

\setlength{\parskip}{2em}


\ifLuaTeX
  \usepackage{selnolig}  % disable illegal ligatures
\fi
\usepackage[]{natbib}
\bibliographystyle{apalike}

\title{Financial Modeling Analysis}
\author{Xie Zejian}
\date{2021-11-08}

\usepackage{amsthm}
\newtheorem{theorem}{Theorem}[chapter]
\newtheorem{lemma}{Lemma}[chapter]
\newtheorem{corollary}{Corollary}[chapter]
\newtheorem{proposition}{Proposition}[chapter]
\newtheorem{conjecture}{Conjecture}[chapter]
\theoremstyle{definition}
\newtheorem{definition}{Definition}[chapter]
\theoremstyle{definition}
\newtheorem{example}{Example}[chapter]
\theoremstyle{definition}
\newtheorem{exercise}{Exercise}[chapter]
\theoremstyle{definition}
\newtheorem{hypothesis}{Hypothesis}[chapter]
\theoremstyle{remark}
\newtheorem*{remark}{Remark}
\newtheorem*{solution}{Solution}
\begin{document}
\maketitle

{
\setcounter{tocdepth}{1}
\tableofcontents
}
\hypertarget{test}{%
\chapter{TEST}\label{test}}

sss

\hypertarget{single-period-model}{%
\chapter{Single Period Model}\label{single-period-model}}

\hypertarget{simple-example}{%
\section{Simple Example}\label{simple-example}}

Suppose one buy a call option \(C\) which payoff \(V_1=(S_1-K)^{+}\) at time \(1\), \(S_1\)
can be \(uS_0>K\) or \(dS_0<K\) determined in probability space \((\left\{ H,T \right\},2^{\left\{ H,T \right\}},\mathop{{}\mathbb{P}})\).
To replicate such option, we construct our portfolio by buying
\(\Delta_0\) stock and investing remaining in risk-free asset at return \(r\):
\[
(V_0-\Delta_0S_0)(1+r)+\Delta_0S_1=(S_1-K)^{+}
\]
solve:
\[
V_0=\frac{p V_1(H)+qV_1(T)}{1+r},\Delta_0= \frac{V_1(H)-V_1(T)}{S_1(H)-S_1(T)}
\]
where \(p=\frac{1+r-d}{u-d},q=1-p\).

\(p\) and \(q\) can be then seen as probability assigned to \(\mathop{{}\mathbb{P}}\left\{ H \right\}\) and \(\mathop{{}\mathbb{P}}\left\{ T \right\}\). Then \(V_0\) is just the discounted of expected value of such option, such measure \(\mathop{{}\mathbb{P}}\) is called \textbf{risk-neutral}.

\hypertarget{state-price}{%
\section{State price}\label{state-price}}

Suppose \(\bm{S_0}\in \mathbb{R}^{N}\) is price of \(N\) stocks at time \(0\),
and \(\bm{D}\in \mathbb{R}^{N\times n}\) is their price at time \(t\) for \(n\) states.
For any portfolio \(\bm{\theta}\in \mathbb{R}^{N}\), it cost \(\bm{S_0'\theta}\) and
its value is \(\bm{D'\theta}\in \mathbb{R}^{n}\) for all \(n\) states.

An arbitrage is then defined as a portfolio \(\theta, s.t.\) \(\bm{S'\theta}\) have different sign with \(\bm{D'\theta}\).

\begin{definition}[State price]
A state price vector is \(\bm{\psi} \in \mathbb{R}_{++}^{n}\) \(s.t.\) \(\bm{S_0}=\bm{D\psi}\).
\end{definition}

To justify the name of state price, suppose we want to ``bet'' the state of market, \(i.e.\), we would like earning \(\bm{1}_{state=i}\), then our portfolio supposed to be \(\bm{D'\theta}=\bm{e_i}\), and it cost
\[
\bm{S_0'\theta}=\bm{\psi'D'\theta=\psi'e_i=}\psi_i
\]
so the coordinate of \(\bm{\psi}\) is the price of ``betting'' a state.

\begin{theorem}[]
There is no arbitrage iff there is a state price vector.
\end{theorem}

\begin{theorem}[Separating Hyperplane Theorem]
\protect\hypertarget{thm:SPH}{}\label{thm:SPH}Suppose \(M\) and \(K\) are closed convex cones in \(\mathbb{R}^{d}\) that \(M \cap K=\left\{ \bm{0} \right\}\), if \(K\) isn't a liner space, then there is a nonzero linear \(f\) separated them, \(i.e.\), \(f(x)<f(y)\) for any \(x\in M\) and \(y \in K- \left\{ \bm{0} \right\}\).
\end{theorem}

\begin{theorem}[Riesz Representation Theorem]
\protect\hypertarget{thm:riesz}{}\label{thm:riesz}Any continuous linear function \(f\) on Hilbert space \(\mathcal{H}\) can be written as \(f(x)=\left\langle x,v \right\rangle\) for some \(v \in \mathcal{H}\).
\end{theorem}

\begin{proof}
Let \(M=\left\{ (-\bm{S_0'\theta},\bm{D'\theta}):\theta \in \mathbb{R}^{N} \right\}\) and \(K=\mathbb{R}_{+}\times \mathbb{R}_{+}^{n}\). Then there is no arbitrage iff \(K\cap M=\left\{ \bm{0} \right\}\).

\(\implies\), let \(f\) be the functional in theorem \ref{thm:SPH}, note \(M\) is a linear space, \(f\) should be vanish on \(M\), \(i.e.\), \(f(x)=0,\forall x \in M\), otherwise, fix \(f(y)>0\) for \(y\in K-\left\{ 0 \right\}\), we can find \(\lambda \in \mathbb{R}\) \(s.t.\) \(\lambda f(x)=f(\lambda x)>f(y)\).

Then by theorem \ref{thm:riesz}, we have \(f(x)=x'v\) for some \(v\), write \(v=(\alpha,\phi)\) where \(\alpha\in \mathbb{R}\) and \(\phi \in \mathbb{R}^{n}\). Since \(f(x)>0\) for nonzero \(x \in K\), \(\alpha\) and \(\phi\) should strictly positive, then
\[
- \alpha \bm{S_0'\theta}+\bm{\phi'D'\theta}=0
\]
which implies \(-\alpha \bm{S_0}+\bm{D\phi}=\bm{0}\) and thus \(\frac{\phi}{\alpha}\) is a state price vector as required.

\(\impliedby\), Suppose \((-\bm{S_0'\theta},\bm{D'\theta}) \in K\), then, \(\bm{\psi'D'\theta}\le 0\) and \(\bm{D'\theta}\), which contrast to \(\psi \gg 0\).
\end{proof}

\begin{exercise}[]
1
\end{exercise}

\begin{solution}
Given above.
\end{solution}

\begin{exercise}[]
2
\end{exercise}

\begin{solution}
Setup:
\[
\bm{S_0}=(1,S_0)', \bm{D}=\begin{bmatrix}
                              1+r   & 1+r   \\
                              uS_0  & dS_0  \\
                          \end{bmatrix}
\]
then the state price should be
\[
\bm{\psi}=\bm{D^{-1}S_0}=[(-u+1+r)S_0,(d-1-r)S_0]'\gg 0
\]
then claim follows.
\end{solution}

\begin{exercise}[]
3
\end{exercise}

\begin{solution}
No.~Let \(\psi=(\frac{1}{3},\frac{1}{3})\), it's a state price vector.
\end{solution}

\begin{exercise}[]
4
\end{exercise}

\begin{solution}
Note column space of \(D\) is just \(\left\{ \lambda\cdot(1,2,3)': \lambda \in \mathbb{R} \right\}\), which excluded \(\overline{q}\), therefore there is no state price vector and thus arbitrage exists.
\end{solution}

\hypertarget{risk-neutral-probability}{%
\subsection{Risk-neutral probability}\label{risk-neutral-probability}}

If \(\bm{p}>0\) and \(\bm{e'p}=1\), we can view \(\bm{p}\in \mathbb{R}^{n}\) as a probability vector represent each state, as there is only \(n\) states, we can use it to represent probability measure. Then
\[
\mathop{{}\mathbb{E}}_{}S_T=\bm{Dp}
\]
take \(\bm{p=\frac{\psi}{e'\psi}}\). Then
\[
\mathop{{}\mathbb{E}}_{}S_T=\bm{\frac{D\psi}{e'\psi}}=\bm{\frac{S_0}{e'\psi}}
\]
where \(\bm{e'\psi}\) is the discount on riskless borrowing. To confirm this, suppose the market allow positive riskless borrowing and we replicate it by investing a portfolio \(\bm{\theta}\) for which
\[
\bm{D'\theta=e}
\]
and \(\bm{\theta}\) cost \(\bm{S_0'\theta=\psi'D'\theta=e'\psi}\). That is \(\bm{\psi_0}=\bm{e'\theta}\) is the riskless discount, \(\text{ i.e. } \frac{1}{\psi_0}=e^{rT}\).

If probability vector \(\bm{p}\) also let \(\mathop{{}\mathbb{E}}_{}S_{T}\) have the same value, we said it's a risk-neutral probability measure.

A claim \(C\in \mathbb{R}^{n}\) and it's said to be attainable or can be hedged if there is a portfolio \(\theta\) that \(\bm{D'\theta}=C\).

\begin{theorem}[]
With absence of arbitrage, the price of an attainable claim \(C\in \mathbb{R}^{n}\) is \(\bm{e'\psi}\mathop{{}\mathbb{E}}_{}C\) if \(\bm{S_0}=\bm{e'\psi}\mathop{{}\mathbb{E}}_{}S_T\) for some probability measure \(\bm{\psi}\).
\end{theorem}

\begin{proof}
Suppose \(\bm{D'\theta=C}\), then its price should be \(\bm{\theta'S_0}\)
\[
\mathop{{}\mathbb{E}}_{}C=\mathop{{}\mathbb{E}}_{}\bm{D'\theta}=\frac{\bm{\theta'D\psi}}{\bm{e'\psi}}=\bm{\theta' S_0}
\]
\end{proof}

A market is said to be complete if every claim \(C\) is attainable.

\begin{theorem}[]
The market in our setting is complete iff \(N\ge n\) and \(\bm{D}\) have full column rank.
\end{theorem}

\begin{proof}
Completeness is precisely equivalent row space \(\mathcal{C}(\bm{D'})=\mathbb{R}^{n}\) and then claim follows.
\end{proof}

In complete market, risk-neutral measure is unique.

\hypertarget{optimality-and-asset-pricing}{%
\section{Optimality and Asset Pricing}\label{optimality-and-asset-pricing}}

Suppose the market \((\bm{D,S_0})\) is given, an agent is defined by an utility function \(U:\mathbb{R}^{n}\to \mathbb{R}\) and an endowment \(\bm{\varepsilon}\in \mathbb{R}^{n}_{+}\). Our optimal target is
\[
\max _{\theta \in A} U(\bm{\varepsilon+D'\theta}) 
\]
where
\[
A=\left\{ \bm{S_0'\theta}\le 0, \bm{\varepsilon+D'\theta \ge  0} \right\}
\]
and we assume there is \(\bm{\theta _0} \text{ s.t. } \bm{D'\theta_0}>0\), that along with absence of arbitrage implies the optimal \(\theta^*\) satisfy \(\bm{S_0'\theta}=0\), otherwise we can invest some on \(\bm{\theta_0}\) and get a better portfolio.

Note \(A=\left\{ \bm{S_0'\theta}= 0, \bm{\varepsilon+D'\theta \ge 0} \right\}\) is closed and bounded if there is no arbitrage and assume \(U\) is continuous, we have

\begin{proposition}[]
The optimal problem has solution iff there is no arbitrage.
\end{proposition}

\begin{theorem}[]
\protect\hypertarget{thm:utility-state}{}\label{thm:utility-state}If in optimal solution \(\bm{\theta^{*}}\), \(\bm{c^{*}}=\bm{\varepsilon+D'\theta^{*}}\gg 0\), \(\nabla U\gg 0\) at \(\bm{c}^{*}\),
There exist \(\lambda>0 \text{ s.t. }\) \(\lambda \nabla U(\bm{c}^{*})\) is a state price vector.
\end{theorem}

\begin{proof}
Suppose \(\bm{\theta^{*}}\) is solution, for any portfolio \(\bm{\theta} \text{ s.t. } \bm{S_0'\theta}=0\), if we combine \(\bm{\theta^{*}}\) and \(\bm{\theta}\), utility will be
\[
g(\alpha)=U[\varepsilon+\bm{D'}(\bm{\theta^{*}}+\alpha \bm{\theta})]=U(\bm{c^{*}}+\alpha \bm{D'\theta})
\]
where \(\bm{c}^{*}=\bm{\varepsilon+D'\theta^{*}}\). As \(\bm{\theta^{*}}\) is the solution, we have FOC on \(\alpha=0\):
\[
g'(0)=\left[ \nabla U(\bm{c}^{*}) \right]'\bm{D'\theta}=[\bm{D}\nabla U(\bm{c}^{*})]'\bm{\theta}=0
\]
that implies \(\bm{D}\nabla U(\bm{c^{*}})=\mu \bm{S}_{0}\) for some \(\mu \in \mathbb{R}\). It's remaining to show that \(\mu>0\). Take \(\bm{\theta_0}\) in assumption, we have
\[
\mu \bm{S_0'\theta_0}=\left[ \nabla U(\bm{c}^{*}) \right]'\bm{D'\theta_0}>0
\]
thus \(\mu>0\) as required.
\end{proof}

Since convex function automatically satisfy SOC, we have

\begin{corollary}[]
\protect\hypertarget{cor:concave-iff}{}\label{cor:concave-iff}If \(U\) is concave and strictly increasing, \(\bm{c^{*}}\gg 0\), then \(\bm{\theta^{*}}\) is the optimal solution iff \(\lambda \nabla U(\bm{c^{*}})\) is a state price vector for some \(\lambda >0\).
\end{corollary}

\hypertarget{expected-utility-function}{%
\subsection{Expected Utility Function}\label{expected-utility-function}}

Now we consider a special case of utility:
\[
U(\bm{c})=\mathop{{}\mathbb{E}}_{}u(c)=\bm{p'u}
\]
where \(u:\mathbb{R}_{+}\to \mathbb{R}\) is concave and increasing and \(\bm{u}=[u(c_1),u(c_2 ),\dots,u(c_n )]'\).

Then we have
\[
\nabla U(\bm{c})=
                \begin{bmatrix}
                     p_1u'(c_1) \\
                     p_2u'(c_2) \\
                     \vdots \\
                     p_nu'(c_n) \\
                 \end{bmatrix}
\]
theorem \ref{thm:utility-state} yields
\[
\bm{S_0}=\bm{D\psi}=\lambda \bm{D} \nabla U(\bm{c}^{*})= \mathop{{}\mathbb{E}}_{} \left[ \lambda S_{T}\cdot u'(c) \right] 
\]
where
\[
\bm{\psi}=\lambda
\begin{bmatrix}
p_1u'(c_1)  \\
p_2u'(c_2)  \\
\vdots \\
p_nu'(c_n)  \\
\end{bmatrix}
\]
thus we can define risk-neutral measure \(\mathbb{Q}\) by \(\bm{\psi}\),\(\bm{S_0}=\psi_0\mathop{{}\mathbb{E}}_{}^{\mathbb{Q}}S_{T}\).

Set \(\bm{\pi}=[\pi_{i}=\psi_{i}/p_{i}]\), then \(\mathop{{}\mathbb{E}}_{}\bm{\pi}=\psi_0\) and \(\bm{S_0}=\mathop{{}\mathbb{E}}_{}S_{T}\pi\)

\begin{remark}
We use boldface to refer the vector representation of a random variable.
\end{remark}

In fact, for any attainable claim \(C\), we have

\begin{theorem}[]
The price of \(C\) is given by \(\mathop{{}\mathbb{E}}_{}\pi \mathop{{}\mathbb{E}}_{}^{Q}C\)
\end{theorem}

\begin{proof}
Suppose \(C\) can be attain by \(\theta\), then
\[
S_0'\theta=[\mathop{{}\mathbb{E}}_{}S_{T}\pi]'\theta=\mathop{{}\mathbb{E}}_{}\pi S_{T}'\theta=\mathop{{}\mathbb{E}}_{}\pi C=\psi_0 \mathop{{}\mathbb{E}}_{}^{Q}C=\mathop{{}\mathbb{E}}_{}\pi\mathop{{}\mathbb{E}}_{}^{Q}C
\]
\end{proof}

\hypertarget{equilibrium}{%
\subsection{Equilibrium}\label{equilibrium}}

\begin{definition}[]
Equilibrium is a pair \((\bm{\theta_{i}})_{i\le m},q\), where \(\bm{\theta}_{i}\) maximize each one's utility and \(\sum_{}^{}\bm{\theta}_{i}=\bm{0}\).
\end{definition}

If each person invest \(\bm{\theta}_{i}\), then we allocate \(\bm{c}_{i}=\bm{\varepsilon}_{i}+\bm{S_{T}'\theta_{i}}\) for each one, and therefore we can define Pareto optimal etc.

\begin{theorem}[The First Welfare Theorem]
Under complete market, equilibrium is Pareto optimal.
\end{theorem}

\hypertarget{discrete-martingale}{%
\section{Discrete Martingale}\label{discrete-martingale}}

Under risk-neutral measure, for each time \(k\), we have
\[
\psi_{0}\mathop{{}\mathbb{E}}_{k}{S}_{k+1}=S_{k}
\]
thus if we define the discounted stock price as \(\widetilde{S}_{k}=\psi_0^{k} S_{k}\)(where each \(\psi_0\) can be vary by time so it should be \(\prod_{i=1} ^{k} \psi_{0,k}\) but we abuse notation here), then \(\widetilde{S}\) became a martingale. That implies the discounted claim \(\widetilde{V}\) is also a martingale.

Now let \((\varphi_n,\psi_n)\) be the amount of stock and bound at time \(n\), the portfolio we holding value:
\[
V_n=\varphi_{n}S_n+\psi_{n}B_n
\]
since the portfolio is self-financing, the should equal to the value at the start of \(n+1\)(when the price is remain the same):
\[
V_n=\varphi_{n+1}S_n+\psi_{n+1}B_n
\]
Put them together and take discount, we have
\[
\widetilde{V}_n=\varphi_{n+1}\widetilde{S}_{n}+\psi_{n+1}=\varphi_{n}\widetilde{S}_{n}+\psi_{n}
\]
we have
\[
\widetilde{V}_{n+1}-\widetilde{V}_n=\varphi_{n+1}(\widetilde{S}_{n+1}-\widetilde{S}_n)
\]
By induction we have
\[
\widetilde{V}_n=V_0+\sum_{i=0}^{n-1}\varphi_{i+1}(\widetilde{S}_{i+1}-\widetilde{S}_{i})
\]
that is a martingale by invoking following lemma:

\begin{lemma}[]
Suppose process \(X=\left\{ X_{t} \right\}_{t\in \mathbb{N}}\) is adapted to \(\mathbb{F}=\left\{ \mathcal{F}_t \right\}_{t \in \mathbb{N}}\) and \(\left\{ \varphi_i \right\}_{i \in \mathbb{N^{+}}}\) is \(\mathbb{F}\)-predictable. Then
\[
\left\{ Z_n\triangleq Z_0+\sum_{i=0}^{n-1}\varphi_{i+1}(X_{i+1}-X_i) \right\}_{n \in \mathbb{N}}
\]
is a martingale if so is \(X\).
\end{lemma}

Now we turn to stocks market consist \(N\) stocks, construct sample space as \(\Omega\) be all possible path \((\mathbb{R}^{N})^{T}\). The absence of arbitrage give a risk-neutral measure for which
\[
S_{t-1}=\psi_{0}\mathop{{}\mathbb{E}}_{S_{t-1}}S_{t}
\]
and as before, take discount \(\widetilde{S}_{t}=\psi_{0}^{t}S_{t}\) we have a martingale. And we claim that absence of martingale is equivalent to existence of risk-neutral measure.

  \bibliography{book.bib,packages.bib}

\end{document}
